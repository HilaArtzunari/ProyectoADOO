% \IUref{IUAdmPS}{Administrar Planta de Selección}
% \IUref{IUModPS}{Modificar Planta de Selección}
% \IUref{IUEliPS}{Eliminar Planta de Selección}

% 


% Copie este bloque por cada caso de uso:
%-------------------------------------- COMIENZA descripción del caso de uso.

%\begin{UseCase}[archivo de imágen]{UCX}{Nombre del Caso de uso}{
	\begin{UseCase}{CU6}{Registrar ventas  de productos}
{
		La caja se encargará de registrar los pagos realizados por los clientes cuando se compre un medicamento o un servicio como consultas médicas, generación de certificados médicos o chequeos.
}
		\UCitem{Versión}{0.1}
		\UCitem{Actor}{Cajero}
		\UCitem{Propósito}{El registro de pago permite hacer el corte de caja al final del día y saber cuántas entradas de dinero hubieron.}
		\UCitem{Entradas}{
				- Nombre de los productos comprados: cadena de carácteres-
				- Cantidad de cada producto: número entero positivo.
		}
		\UCitem{origen}{Teclado: Productos comprados y la cantidad de éstos.}
		\UCitem{Salidas}{
					- Un mensaje “Registro de venta exitoso.”
					
					- Un botón de Aceptar.
}
		\UCitem{Destino}{El panel actual donde se registró la venta.}
		\UCitem{Precondiciones}{
					- Los productos ingresados deben estar registrados con anterioridad.
					
					- La cantidad ingresada por cada uno debe ser un número entero positivo menor o igual a la cantidad registrada del producto en el almacén.
		}
		\UCitem{Postcondiciones}{
					- Se registra la venta en la base de datos.
					
					- Se actualiza la cantidad de productos en el stock.
					
					- Se vuelve a la pantalla de inicio.
		}
		\UCitem{Errores}{
					-	El actor introduce un número negativo en la cantidad de producto, o bien un número superior a la cantidad en el almacén.
					
					- El actor coloca un nombre de producto que no está registrado.
}
		\UCitem{Tipo de ejecución}{Secundario, viene de CU0: Login}
		\UCitem{Autor}{Gabriela Moreno González}
		\UCitem{Revisó}{Erick Rubén Reynoso Rodríguez}
	\end{UseCase}
	
\begin{UCtrayectoria}{Principal}
	\UCpaso	El actor ingresa el nombre del producto comprado.
	\UCpaso El actor solicita al sistema que busque dicho producto. \Trayref{A}
	\UCpaso El sistema encuentra dicho producto y solicita la cantidad vendida.
	\UCpaso El actor ingresa la cantidad vendida de dicho producto y acepta. \Trayref{B}
	\UCpaso El actor solicita la generación del ticket.
	\UCpaso El sistema actualiza la base de datos registrando la venta.
	\UCpaso El sistema genera el ticket.
	\UCpaso Fin
\end{UCtrayectoria}
		
\begin{UCtrayectoriaA}{A}{El nombre del producto ingresado no está registrado.}
			\UCpaso El sistema muestra un mensaje 'El nombre del producto ingresado está mal escrito o no está registrado'
			\UCpaso El actor oprime el botón de Aceptar.
			\UCpaso[] Termina el caso de uso.
\end{UCtrayectoriaA}

\begin{UCtrayectoriaA}{B}{La cantidad ingresada supera a la cantidad en el almacén.}
			\UCpaso El sistema muestra un mensaje 'La cantidad que deseas vender supera a la cantidad disponible.'
			\UCpaso El actor oprime el botón de Aceptar.
			\UCpaso[] Termina el caso de uso.
\end{UCtrayectoriaA}
%-------------------------------------- TERMINA descripción del caso de uso.