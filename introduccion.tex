En este documento se presentará el análisis, diseño, construcción y las pruebas de nuestro Sistema Administrador de Clínicas, así como la metodología usada y las etapas que harán que el sistema vaya creciendo y permita resolver las problemáticas que actualmente enfrenta la empresa.

El presente documento se encuentra divido por 7 bloques: Introducción, Análisis de problema, Propuesta de solución, Modelo de Negocios, Modelo de despliegue del sistema, Modelo de comportamiento y modelo de la iteración.

En la introducción que en este momento lee se da una pequeña reseña del trabajo, así como los acrónimos, abreviaturas y referencias bibliográficas que se han consultado con el fin de diseñar este documento de la forma más amigable y entendible para el lector.

En la sección de Análisis del problema se tratará el contexto del sistema, el cual define cómo se mueve la empresa actualmente; los procesos actuales, que describe quiénes y cuál es su puesto dentro del sistema; los problema identificados, que son las razones que lleva a la empresa a solicitar un sistema y que no permiten que funcione de manera eficaz; y, finalmente, las propuestas de solución, que son las alternativas que se podrían aplicar para resolver los problemas identificados, de esas alternativas de solución se seleccionará la que mejor resuelva el problema y cumpla los requisitos que la empresa solicita.

La propuesta de solución se desglosará planteando primeramente los objetivos: un objetivo general y varios partículares, que serán las metas que queremos lograr con nuestro sistema; y se tratará el modelo de despliegue abarcando los requerimientos no funcionales, el modelado de dicho despliegue y las especificaciones de la plataforma.

El modelo de negocios nos permitirá definir cómo trabaja la empresa actualmente y si en algún punto el software que se planea implementar podría llegar a cambiar la forma en la que se mueve la empresa. Primeramente se creará un glosario de términos para entender la jerga de los empleados, se tratarán los procesos ajustados, así como los procesos actuales, la descripción de atributos y finalmente las reglas del negocio, que son las principales y son las que rigen todo sistema.

El modelo de despliegue del sistema en una sección encargada de mostrar cómo nuestro sistema se va desarrollando a través del tiempo y cómo está estructurado.

El modelo de comportamiento describe qué funcionalidades tiene el sistema y cómo debe reaccionar ante diversos eventos que genere el usuario, describiendo sus atributos y cómo se va a mover el sistema en caso de entradas no esperadas.

El modelo de iteraciones presentará la descripción completa de la sinterfaces de usuario y cómo éste puede manipularlas e interactuar con ellas para obtener un resultado definido.

Este documento va dirigido al profesor Ulises Vélez Saldaña, profesor de la Escuela Superior de Cómputo del Instituto Politécnico Nacional como un proyecto de desarrollo de software.

Este documento será realizado por 'The Dream Team', conformado por:

- Martínez Vilchis Juan Moisés.

- Moreno González Gabriela.

- Pérez Montiel Ulises.

- Reynoso Rodríguez Erick Rubén.

- Saldaña Campos Sebastián.

Realizado en la Escuela Superior de Cómputo del Instituto Politécnico Nacional, mediante una organización secuencial de las partes que se irán cubriendo del documento y de la presentación final.

%--------------------------------------------------
\section{Propósito}
El propósito de nuestro sistema es resolver las problemáticas que presenta la forma en la que se trata el negocio de la clínica actualmente mediante el uso de software desarrollado y de tecnologías web.

%--------------------------------------------------
\section{Alcance}
Nuestro proyecto planea cubrir todos los requerimientos funcionales y no funcionales que serán pplanteados y analizados a lo largo del proyecto para resolver las problemáticas principales. Se espera que los problemas secundarios sean tratados en tiempo posterior a la entrega de este proyecto.

%--------------------------------------------------
\section{Definiciones, acrónimos y abreviaturas}

%--------------------------------------------------
\section{Referencias}

* Bruegge, Bernd, y Allen H. Dutoit. Object-Oriented Software Engineering Conquering Complex and Changing Systems. 1 ed. Pittsburgh, USA: Prentice Hall, 1999. Impreso. \\

* Docherty, Mike O'. Object-Oriented Analysis and Design Understanding System Development with UML 2.0. 1 ed. England: Joh Willey and Sons, Ltd, 2005. Impreso. \\

* Booch, Grady, Robert A. Maksimchuk, Michael W. Engle, Bobbi J. Young, Jim Conallen, y Kelli A.  Houston. Object-Oriented Analysis and Design with Applications. 3 ed. Mexico City: Series Editors, 2000. Impreso. \\

* Bruegge, Bernd, y Allen H. Dutoit. Object-Oriented Software Engineering Using UML, Patterns, and Java™. 3 ed. Pittsburgh, PA, United States: Prentice Hall, 2010. Impreso.

%--------------------------------------------------
\section{Contenido y organización}
