% \IUref{IUAdmPS}{Administrar Planta de Selección}
% \IUref{IUModPS}{Modificar Planta de Selección}
% \IUref{IUEliPS}{Eliminar Planta de Selección}

% 


% Copie este bloque por cada caso de uso:
%-------------------------------------- COMIENZA descripción del caso de uso.

%\begin{UseCase}[archivo de imágen]{UCX}{Nombre del Caso de uso}{
	\begin{UseCase}{CU1}{Consultar Productos}
{
		Cuando se requiera realizar una venta o verificar si hay en existencia algún producto, se podrá realizar una consulta por el nombre o ingrediente activo. El sistema deberá mostrar la información de dichos productos.
}
		\UCitem{Versión}{0.1}
		\UCitem{Actor}{Cajero}
		\UCitem{Propósito}{El cajero sepa si existe el producto para informarle a los clientes si hay existencia del producto que se desea comprar}
		\UCitem{Entradas}{Nombre del Producto/Nombre ingrediente activo: Cadena de texto (20 digitos)}
		\UCitem{origen}{Teclado: Nombre del Producto/Nombre ingrediente activo}
		\UCitem{Salidas}{
					Información del producto que coincida con el nombre ingresado o ingrediente activo y los similare de dicho producto seleccionado, donde se muestre:
					
					- Número de existencias
					
					-	Precio
					
					-	Descripción
					
					-	Nombre del Producto
}
		\UCitem{Destino}{Una pantalla donde se muestre la información del producto}
		\UCitem{Precondiciones}{Debe existir en la farmacia, el producto que se desea o similares.}
		\UCitem{Postcondiciones}{Ninguna}
		\UCitem{Errores:}{
					-	El actor introduce mal el nombre o ingrediente que se desea.
					
					-	No hay conexión con la base de datos.
					
					-	No coincida el nombre del producto o el ingrediente con el ya registrado.
}
		\UCitem{Tipo de ejecución}{Secundario, viene de CU0: Login}
		\UCitem{Autor}{Juan Moisés Martínez Vilchis.}
		\UCitem{Revisó}{Sebastián Saldaña Campos.}
	\end{UseCase}
	
\begin{UCtrayectoria}{Principal}
	\UCpaso[\UCactor] En la pantalla PCa1, en la barra de búsqueda, se solicita realizar una búsqueda de algún producto tecleando en la barra de texto el nombre del producto o del ingrediente activo que se desea.
	\UCpaso Realiza una búsqueda, donde coincida el nombre ingresado con alguno ya registrado \Trayref{A}.
	\UCpaso Calcula el numero de exitencias de lo productos registrados.
	\UCpaso Realiza una búsqueda con los productos similares y calcula el numero de existencias registrados.
	\UCpaso Da a conocer el resultado, mostrando el nombre, descripción, precio y número de existencias. 
	\UCpaso Fin
\end{UCtrayectoria}

                 \begin{UCtrayectoriaA}{A}{El Nombre no coincide con alguno ya registrado}
			\UCpaso Realiza una búsqueda, donde coincida el ingrediente activo con alguno ya registrado.
                        \UCpaso fin.
		\end{UCtrayectoriaA}
		
%-------------------------------------- TERMINA descripción del caso de uso.
