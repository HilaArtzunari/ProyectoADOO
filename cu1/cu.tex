% \IUref{IUAdmPS}{Administrar Planta de Selección}
% \IUref{IUModPS}{Modificar Planta de Selección}
% \IUref{IUEliPS}{Eliminar Planta de Selección}

% 


% Copie este bloque por cada caso de uso:
%-------------------------------------- COMIENZA descripción del caso de uso.

%\begin{UseCase}[archivo de imágen]{UCX}{Nombre del Caso de uso}{
	\begin{UseCase}{CU1}{Consultar Productos}
{
		Cuando se requiera realizar una venta o verificar si hay en existencia algún producto, se podrá realizar una consulta por el nombre o en general. El sistema deberá mostrar la información de dichos productos.
}
		\UCitem{Versión}{0.1}
		\UCitem{Actor}{Cajero}
		\UCitem{Propósito}{El cajero sepa si existe el producto para informarle a los clientes si hay existencia del producto que se desea comprar}
		\UCitem{Entradas}{Nombre del Producto: Cadena de texto (20 digitos)}
		\UCitem{origen}{Teclado: Nombre del Producto}
		\UCitem{Salidas}{
					Información del producto que coincida con el nombre ingresado, donde se muestre:
					
					- Número de existencias
					
					-	Precio
					
					-	Descripción
					
					-	Nombre del Producto
					
					-	Un botón si se desea comprar el producto
}
		\UCitem{Destino}{Un panel donde se muestre la información del producto}
		\UCitem{Precondiciones}{Debe existir en la farmacia, el producto que se desea.}
		\UCitem{Postcondiciones}{Observar información acerca del producto, Informar al cliente la información, Realizar la venta del producto.}
		\UCitem{Errores:}{
					-	El actor introduce mal el nombre que se desea.
					
					-	No hay conexión con la base de datos.
					
					-	No coincida el nombre del producto con el ya registrado.
}
		\UCitem{Tipo de ejecución}{Secundario, viene de CU0: Login}
		\UCitem{Autor}{Juan Moisés Martínez Vilchis.}
		\UCitem{Revisó}{Sebastián Saldaña Campos.}
	\end{UseCase}
	
\begin{UCtrayectoria}{Principal}
	\UCpaso	El actor solicita realizar una búsqueda de algún producto tecleando en la barra de texto el nombre que se desea.
	\UCpaso El sistema realiza una búsqueda en en la base de datos, en la tabla de “Medicamentos”, donde coincida el nombre ingresado con alguno ya registrado.
	\UCpaso El sistema verifica si hay existencias el producto introducido
	\UCpaso El sistema realiza una búsqueda con los productos similares
	\UCpaso El sistema da a conocer el resultado, mostrando el nombre, descripción, precio y número de existencias. 
	\UCpaso Fin
\end{UCtrayectoria}
		
%-------------------------------------- TERMINA descripción del caso de uso.