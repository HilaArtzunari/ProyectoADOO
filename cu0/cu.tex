% \IUref{IUAdmPS}{Administrar Planta de Selección}
% \IUref{IUModPS}{Modificar Planta de Selección}
% \IUref{IUEliPS}{Eliminar Planta de Selección}

% 


% Copie este bloque por cada caso de uso:
%-------------------------------------- COMIENZA descripción del caso de uso.

%\begin{UseCase}[archivo de imágen]{UCX}{Nombre del Caso de uso}{
	\begin{UseCase}{CU0}{Login}
{
		El usuario podrá acceder a su cuenta de usuario desde el sitio web mediante el ingreso de su correo y su contraseña.
}
		\UCitem{Versión}{0.1}
		\UCitem{Actor}{Cliente}
		\UCitem{Propósito}{Poder solicitar una cita}
		\UCitem{Entradas}{
			-	Correo: Cadena de caracteres con el formato: example@dominio.com
			
			-	Contraseña: Cadena de caracteres de mínimo 6 y máximo 20.
}
		\UCitem{origen}{Todas las entradas del teclado}
		\UCitem{Salidas}{
				-	Mensaje de confirmación: “Bienvenido nombreusuario”
}
		\UCitem{Destino}{Pantalla del actor}
		\UCitem{Precondiciones}{
		-	Debe existir una cuenta de usuario que tenga asignado el correo ingresado.
		
		-	La contraseña debe coincidir con la contraseña asignada a la cuenta de usuario a la que corresponda el correo ingresado.
		
		-	Conexión a internet.
}
		\UCitem{Postcondiciones}{
		-	Se inicia la sesión.
		
		-	Se cargan los datos del usuario.
		
		-	Se actualiza la página principal.
}
		\UCitem{Errores:}{
					-	Si el correo no está registrado se informa al usuario.
					
					-	Si la contraseña no coincide se informa al usuario.
					
					-	Si la contraseña tiene menos de 6 caracteres o más de 20 no se permite iniciar sesión.
					
					-	Si ya se ha iniciado sesión con esa cuenta se bloquean ambas cuentas y se solicita que pida un desbloqueo en la consultoría.
}
		\UCitem{Tipo de ejecución}{Primario}
		\UCitem{Autor}{Gabriela Moreno González}
		\UCitem{Revisó}{Sebastián Saldaña Campos.}
	\end{UseCase}
	
\begin{UCtrayectoria}{Principal}
	\UCpaso	El actor ingresa su correo y su contraseña.
	\UCpaso El actor oprime el botón de 'Iniciar sesión.'
	\UCpaso El sistema busca los datos que correspondan al correo y a la contraseña. \Trayref{A}
	\UCpaso El sistema muestra el mensaje de 'Bienvenido nombreactor'
	\UCpaso Fin
\end{UCtrayectoria}

\begin{UCtrayectoriaA}{A}{El correo o contraseña no coinciden.}
			\UCpaso El sistema muestra un mensaje 'Correo o contraseña incorrectos'
			\UCpaso El actor oprime el botón de Aceptar.
			\UCpaso[] Termina el caso de uso.
\end{UCtrayectoriaA}
		
%-------------------------------------- TERMINA descripción del caso de uso.