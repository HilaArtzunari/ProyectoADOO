% \IUref{IUAdmPS}{Administrar Planta de Selección}
% \IUref{IUModPS}{Modificar Planta de Selección}
% \IUref{IUEliPS}{Eliminar Planta de Selección}

% 


% Copie este bloque por cada caso de uso:
%-------------------------------------- COMIENZA descripción del caso de uso.

%\begin{UseCase}[archivo de imágen]{UCX}{Nombre del Caso de uso}{
	\begin{UseCase}{CU10}{Dar de alta paciente}
{
		Cuando un paciente tome una consulta por primera vez, el médico registra al paciente en el sistema proporcionando sus datos. El sistema guardará sus datos.
}
		\UCitem{Versión}{0.1}
		\UCitem{Actor}{Médico}
		\UCitem{Propósito}{Que el médico que lo atienda pueda confirmar las citas que se han llevado a cabo con dicho paciente, como el crear, modificar el expediente}
		\UCitem{Entradas}{
		-	Nombre: String
		
		-	Apellido Paterno: String
				
				-	Apellido Materno: String
				
				-	Fecha de Nacimiento:
				
				-	Día: int
				
				-	Mes: int
				
				-	Año: int
				
				-	CURP: String de 18 Digitos
				
				-	Dirección:
				
				-	Calle: String
				
				-	No Exterior: int
				
				-	No Interior: int
				
				-	Colonia: String
				
				-	Codigo Postal: int de 5 Digitos
				
				-	Delegación o Municipio: String
				
				-	Estado: String
				
				-	País: String
				
				-	Teléfono: 
				
				-	Telefono: String
				
				-	Tipo: String
				
				-	Teléfono de Emergencia:
				
				-	Telefono: String
				
				-	Tipo: String
				
				-	Dueño: String
		}
		\UCitem{origen}{Teclado.}
		\UCitem{Salidas}{
					-	Mensaje: Registro Exitoso
}
		\UCitem{Destino}{Mensaje en Pantalla}
		\UCitem{Precondiciones}{
					-	Que no esté registrado otro paciente con la misma CURP
		}
		\UCitem{Postcondiciones}{
					-	Agendar Cita
					
					-	Completar Cita
					
					-	Crear Expediente
					
					-	Modificar Expediente
					
					-	Eliminar Paciente
		}
		\UCitem{Errores}{
					-	El médico introduce datos incorrectos
					
					-	No hay conexión con la base de datos
					
					-	Que ya esté registrado otro paciente con el mismo CURP
}
		\UCitem{Tipo de ejecución}{Secundario, viene de CU0: Login}
		\UCitem{Autor}{Erick Rubén Reynoso Rodríguez}
		\UCitem{Revisó}{Sebastián Saldaña Campos}
	\end{UseCase}
	
\begin{UCtrayectoria}{Principal}
	\UCpaso	El actor solicita registrar paciente dando click a “registrar paciente”
	\UCpaso El sistema solicita los datos del paciente a registrar mostrando el formulario “Registrar Paciente”
	\UCpaso El actor introduce los datos del paciente
	\UCpaso El actor presiona click en “Registrar”
	\UCpaso El sistema verifica que no esté registrado el CURP con otro paciente
	\UCpaso El sistema registra al paciente
	\UCpaso El sistema notifica el resultado de la operacion mostrando el mensaje “Registro Exitoso”
	\UCpaso Fin
\end{UCtrayectoria}

%-------------------------------------- TERMINA descripción del caso de uso.