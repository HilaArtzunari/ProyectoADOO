% \IUref{IUAdmPS}{Administrar Planta de Selección}
% \IUref{IUModPS}{Modificar Planta de Selección}
% \IUref{IUEliPS}{Eliminar Planta de Selección}

% 


% Copie este bloque por cada caso de uso:
%-------------------------------------- COMIENZA descripción del caso de uso.

%\begin{UseCase}[archivo de imágen]{UCX}{Nombre del Caso de uso}{
	\begin{UseCase}{CU2}{Consultar productos por similar}
{
		Cuando se requiera realizar una venta o verificar si hay en existencia algún producto, se podrá realizar una consulta por los similares del producto deseado. El sistema deberá mostrar la información de dichos productos.
}
		\UCitem{Versión}{0.1}
		\UCitem{Actor}{Cajero}
		\UCitem{Propósito}{El cajero sepa si existe el producto para informarle a los clientes si hay existencia del producto que se desea comprar}
		\UCitem{Entradas}{Nombre del Producto: Cadena de texto (20 digitos)}
		\UCitem{origen}{Teclado: Nombre del Producto}
		\UCitem{Salidas}{
					Información de los productos que sean similares al producto deseado, donde se muestre:

						-	Número de existencias

						-	Precio

						-	Descripción

						-	Nombre del Producto

						-	Una botón si se desea comprar el producto

}
		\UCitem{Destino}{Un panel donde se muestre la información de los productos}
		\UCitem{Precondiciones}{Deben existir en la farmacia el producto deseado }
		\UCitem{Postcondiciones}{Observar información acerca de los productos, Informar al cliente la información de éstos productos y diferencias entre los mismos y el deseado , Realizar la venta del producto.}
		\UCitem{Errores}{
					-	El actor introduce mal el ingrediente que se desea
					
					-	No hay conexión con la base de datos
					
					-	No coincida el nombre del producto con el ya registrado 
}
		\UCitem{Tipo de ejecución}{Secundario, viene de CU0: Login}
		\UCitem{Autor}{Juan Moisés Martínez Vilchis.}
		\UCitem{Revisó}{Ulises Pérez Montiel.}
	\end{UseCase}
	
\begin{UCtrayectoria}{Principal}
	\UCpaso	El actor solicita realizar una búsqueda de algún producto tecleando en la barra de texto el nombre que se desea.
	\UCpaso El sistema realiza una búsqueda en en la base de datos, en la tabla de “Medicamentos”, donde coincida el nombre ingresado con alguno ya registrado.
	\UCpaso El sistema verifica si hay existencias el producto introducido
	\UCpaso El sistema realiza una búsqueda con los productos similares
	\UCpaso El sistema da a conocer el resultado, mostrando el nombre, descripción, precio y número de existencias. 
	\UCpaso Fin
\end{UCtrayectoria}
		
%-------------------------------------- TERMINA descripción del caso de uso.