% \IUref{IUAdmPS}{Administrar Planta de Selección}
% \IUref{IUModPS}{Modificar Planta de Selección}
% \IUref{IUEliPS}{Eliminar Planta de Selección}

% 


% Copie este bloque por cada caso de uso:
%-------------------------------------- COMIENZA descripción del caso de uso.

%\begin{UseCase}[archivo de imágen]{UCX}{Nombre del Caso de uso}{
	\begin{UseCase}{CU5}{Consultar cita}
{
		La caja podrá consultar las citas que se encuentran agendadas para un día determinado para así poder solicitar el pago al cliente en caso de que éste haya llegado a su cita.
}
		\UCitem{Versión}{0.1}
		\UCitem{Actor}{Cajero}
		\UCitem{Propósito}{Consultar las citas permite llevar un control de aquellas que si se encuentran dadas de alta en el sistema, para que aquellas que no y el cliente desea pagar se le solicite primer registrar su cita.}
		\UCitem{Entradas}{
				- Fecha de la cita: cadena de carácteres con el formato YY/MM/DD
				
				- Hora de la cita: cadena de carácteres con el formato HH:MM
		}
		\UCitem{origen}{Teclado: Fecha y hora de la cita.}
		\UCitem{Salidas}{
					- La información de la fecha, hora, quién solicitó la cita, en qué consultorio y el doctor que atenderá dicha cita.
					
					- Un botón de 'Registrar pago de cita' en caso no haberse pagado aún, o bien un botón de 'Aceptar' en caso de que la cita ya haya sido pagada.
}
		\UCitem{Destino}{El panel actual donde se pidió la consulta de la cita.}
		\UCitem{Precondiciones}{
					- La fecha de la cita debe estar escrita con el formato YY/MM/DD
					
					- La hora de la cita debe estar escrita con el formato HH:MM
		}
		\UCitem{Postcondiciones}{
					- Se consulta la base de datos.
					
					- Se cargan los datos actuales.
					}
		\UCitem{Errores}{
					-	La fecha no se ingresa con el formato YY/MM/DD
					
					- La hora no se ingresa con el formato HH:MM
}
		\UCitem{Tipo de ejecución}{Secundario, viene de CU0: Login}
		\UCitem{Autor}{Gabriela Moreno González}
		\UCitem{Revisó}{Saldaña Campos Sebastián}
	\end{UseCase}
	
\begin{UCtrayectoria}{Principal}
	\UCpaso	El actor ingresa la fecha de la cita.
	\UCpaso El actor ingresa la hora de inicio de la cita. 
	\UCpaso El actor solicita al sistema la información de la cita. \Trayref{A} \Trayref{B}
	\UCpaso El sistema busca en la base de datos la hora y la fecha solicitada. \Trayref{C}
	\UCpaso El sistema muestra la información de la cita.
	\UCpaso El actor oprime el botón 'Aceptar'.
	\UCpaso Fin
\end{UCtrayectoria}
		
\begin{UCtrayectoriaA}{A}{La fecha está escrita con un formato incorrecto.}
			\UCpaso El sistema muestra un mensaje 'Asegúrate de que la fecha esté escrita en formato YY/MM/DD'
			\UCpaso El actor oprime el botón de Aceptar.
			\UCpaso[] Termina el caso de uso.
\end{UCtrayectoriaA}

\begin{UCtrayectoriaA}{B}{La hora está escrita con un formato incorrecto.}
			\UCpaso El sistema muestra un mensaje 'Asegúrate de que la hora esté escrita con formato HH:MM.'
			\UCpaso El actor oprime el botón de Aceptar.
			\UCpaso[] Termina el caso de uso.
\end{UCtrayectoriaA}

\begin{UCtrayectoriaA}{C}{La fecha y hora especificados no existen en la base de datos.}
			\UCpaso El sistema muestra un mensaje 'Fecha y hora ingresados no poseen una cita.'
			\UCpaso El actor oprime el botón de Aceptar.
			\UCpaso[] Termina el caso de uso.
\end{UCtrayectoriaA}
%-------------------------------------- TERMINA descripción del caso de uso.