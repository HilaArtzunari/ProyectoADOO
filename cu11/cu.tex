% \IUref{IUAdmPS}{Administrar Planta de Selección}
% \IUref{IUModPS}{Modificar Planta de Selección}
% \IUref{IUEliPS}{Eliminar Planta de Selección}

% 


% Copie este bloque por cada caso de uso:
%-------------------------------------- COMIENZA descripción del caso de uso.

%\begin{UseCase}[archivo de imágen]{UCX}{Nombre del Caso de uso}{
	\begin{UseCase}{CU11}{Dar de alta un familiar}
{
		Cuando un paciente se tenga registrado se puede dar la opción de registrar a un familiar de dicho paciente
}
		\UCitem{Versión}{0.1}
		\UCitem{Actor}{Médico}
		\UCitem{Propósito}{Los pacientes que tengan a sus familiares registrados les da un mejor expediente médico, por razones hereditarias}
		\UCitem{Entradas}{
		-	Nombre
		
		-	Apellido Paterno
		
		-	Apellido Materno
		
		-	CURP
		
		-	Enfermedades
		}
		\UCitem{origen}{Teclado.}
		\UCitem{Salidas}{
					-	Mensaje: Registro Exitoso
}
		\UCitem{Destino}{Mensaje en Pantalla}
		\UCitem{Precondiciones}{
					-	Tener un paciente registrado

					-	No tener el CURP ya relacionado al paciente
		}
		\UCitem{Postcondiciones}{
					-	Agregar enfermedades hereditarias
		}
		\UCitem{Errores}{
					-	El CURP ya esté relacionado 
}
		\UCitem{Tipo de ejecución}{Secundaria viene de CU15}
		\UCitem{Autor}{Erick Rubén Reynoso Rodríguez}
		\UCitem{Revisó}{Ulises Pérez Montiel}
	\end{UseCase}
	
\begin{UCtrayectoria}{Principal}
	\UCpaso	El actor selecciona agregar datos familiares
	\UCpaso El sistema muestra formulario “agregar familiar”
	\UCpaso El actor introduce los datos del familiar
	\UCpaso El actor le da click en agregar
	\UCpaso El sistema verifica que el CURP no tenga una “relación familiar” con el paciente de la cita seleccionada
	\UCpaso El sistema registra al familiar
	\UCpaso El sistema muestra un mensaje de registro exitoso
	\UCpaso Fin
\end{UCtrayectoria}

%-------------------------------------- TERMINA descripción del caso de uso.