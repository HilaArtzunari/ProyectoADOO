% \IUref{IUAdmPS}{Administrar Planta de Selección}
% \IUref{IUModPS}{Modificar Planta de Selección}
% \IUref{IUEliPS}{Eliminar Planta de Selección}

% 


% Copie este bloque por cada caso de uso:
%-------------------------------------- COMIENZA descripción del caso de uso.

%\begin{UseCase}[archivo de imágen]{UCX}{Nombre del Caso de uso}{
	\begin{UseCase}{CU15}{Generar tratamiento}
{
		Al terminar la consulta el médico le proporcionará al paciente, de manera impresa, un tratamiento que contiene las indicaciones y/o receta médica para tratar su condición. El tratamiento también se almacenará en el expediente del paciente.
}
		\UCitem{Versión}{0.1}
		\UCitem{Actor}{Médico}
		\UCitem{Propósito}{El paciente conozca las acciones que debe tomar para tratar su enfermedad, y sea capaz de comprar los medicamentos necesarios. Se lleve un registro en el expediente de todos los tratamientos que haya llevado el paciente.}
		\UCitem{Entradas}{
		-	Correo electrónico del paciente: String.
		
		-	Tratamiento indicado: String
		}
		\UCitem{origen}{Teclado.}
		\UCitem{Salidas}{
					-	Mensaje de confirmación: “Tratamiento asignado exitosamente”.
					
					-	Copia para paciente: 
					
						- Nombre del Médico
						
						- Nombre del paciente
						
						- Fecha
						
						- Tratamiento indicado por el médico
						
						- Marcador de posición de firma del médico
						
						- Cédula profesional del médico
}
		\UCitem{Destino}{
		-	Mensaje de Confirmación: mensaje en pantalla
		
		-	Copia para paciente: Archivo PDF listo para imprimir
}
		\UCitem{Precondiciones}{
					-	El médico debe haber iniciado sesión en el sistema
					
					-	El paciente debe de estar registrado en el sistema
					
					-	El paciente debe contar con un expediente 
		}
		\UCitem{Postcondiciones}{
					-	El paciente contará con una copia del tratamiento indicado para su consulta
					
					-	El paciente tendrá un comprobante que le permitirá comprar los medicamentos indicados en el tratamiento
					
					-	Se tratamiento será registrado en el expediente del paciente
		}
		\UCitem{Errores}{
					-	El actor introduce información incorrecta
					
					-	El paciente no está registrado en el sistema
					
					-	El paciente no cuenta con un expediente
}
		\UCitem{Tipo de ejecución}{Secundaria viene de CU0}
		\UCitem{Autor}{Sebastián Saldaña Campos}
		\UCitem{Revisó}{Erick Rubén Reynoso Rodríguez}
	\end{UseCase}
	
\begin{UCtrayectoria}{Principal}
	\UCpaso	El actor solicita generar un tratamiento dando click en la opción “Generar Tratamiento”
	\UCpaso El sistema solicita el correo electrónico con el que el paciente creó su cuenta de usuario
	\UCpaso El actor introduce el correo electrónico del paciente
	\UCpaso El actor envía el correo electrónico del paciente al sistema presionando el botón “Buscar”
	\UCpaso El sistema verifica la existencia del usuario buscando su correo electrónico 
	\UCpaso El sistema solicita el tratamiento que se determinó para el paciente
	\UCpaso El sistema registra el tratamiento en el expediente 
	\UCpaso Fin
\end{UCtrayectoria}

%-------------------------------------- TERMINA descripción del caso de uso.